\section{Discussion on Data Usage}

In the appendix we have provided as much of the derived data as possible (the atomic collision data made violable by \citet{schultz2019} and \citet{gharibnejad2019}) with the goal that anyone can use it to estimate their own X-ray flux as long as they have access to an initial JEDI spectrum.
Here we want to layout as clearly as possible how to take an ion flux and produce an X-ray power.

\begin{enumerate}
    \item The first, and arguably most difficult, part is converting the JEDI energy spectrogram into a usable ion flux.
    To be done accurately, this requires knowing the width of each energy bin on JEDI at the time of measurement.
    We have included the energy bin widths in Table \ref{tab:JEDIFlux} that correspond to the data in Figure \ref{fig:JEDIFlux}.
    It is likely the energies bins will be changed and resized, if they have not already.
    \item Once the bin widths are known, one can convert the intensity from counts/steradian/cm$^2$/s/keV to counts/cm$^2$/sec by multiplying each flux intensity by 2$\pi$ and the corresponding energy bin width.
    A second thing to consider is that the first three energy bins cannot distinguish between oxygen and sulfur ions.
    In this study we used an O:S ratio of 2:1 motivated by the likely source of SO$_2$ from Io's volcanoes.
    A different ratio can be used, but those low energies will not affect X-ray production, anyway.
    \item Once an intensity of counts/cm$^2$/s vs. ion energy (in keV/u, not total energy) is obtained, one can multiply the intensity by the ion energy (keV/u) and the X-ray efficiency for each charge state of the ion species at a given ion energy in Appendix \ref{app:Oxy} or \ref{app:Sul}.
    To account for all X-rays, charge exchange and direct excitation need to be considered, in which case the X-ray efficiencies can be summed together.
    This will result in the number of photons/cm$^2$/s produced by each ion charge state and species.
    \item Summing the photon production rate for each charge state together will give the total X-ray production rate  for a given JEDI pass.
    \item Multiplying the photon production rate by the average photon energy, 1.6x10$^{-19}$ J/eV, 10$^6$ $\mu$W/W, and 10$^4$ cm$^2$/m$^2$ will yield the power in $\mu$W/m$^2$.
    In general, the average photon energy is likely between 500-600 eV.
    If sulfur emission is higher than oxygen, then 500 eV is more accurate and if oxygen emission is greater, the average photon energy probably tends closer to 600 eV.
\end{enumerate}

As an example, for the JEDI oxygen measurement discussed in this text, the total photon production and power is calculated at each step in Table \ref{tab:JEDIFlux}.

\begin{table}
    \centering
    \caption{Example of calculating X-ray production rates associated with JEDI oxygen ion flux measurements.}
    \begin{tabular}{r|c|c|c|c}
    \multicolumn{5}{c}{Oxygen} \\ \cline{1-5}
    Energy & JEDI Flux* & Energy Bin & Intensity & Energy  \\
    $\mathrm{[keV]}$ & [c/str/cm$^2$/s/keV] & Width [keV] & [c/cm$^2$/s] & $\mathrm{[keV/u]}$ \\ \cline{1-5}
     171  & 249.9 &   66 & 103631 & 11  \\
     240  & 339.7 &   71 & 151542 & 15  \\
     324  & 279.0 &  105 & 184066 & 20  \\
     477  & 219.8 &  216 & 298306 & 30  \\
     746  & 89.50 &  346 & 194571 & 47  \\
     956  & 43.61 &  251 &  68776 & 60  \\
    1240  & 22.56 &  300 &  42525 & 78  \\
    1930  & 8.687 &  880 &  48032 & 121 \\
    3490  & 3.018 & 2280 &  43235 & 218 \\
    7300  & 0.914 & 5340 &  30667 & 456 \\ \cline{1-5}
    Energy  & X-Ray Efficiency$\dagger$ & X-ray Production \\
    $\mathrm{[keV/u]}$ & [cm$^2$sec]$^{-1}$[keV/u]$^{-1}$ & photons/cm$^2$/s \\ \cline{1-3}
    11    & 0.0000 & 0.000 \\
    15    & 0.0000 & 0.000 \\
    20    & 0.0000 & 0.000 \\
    30    & 0.0000 & 0.000 \\
    47    & 0.0000 & 0.000 \\
    60    & 0.0000 & 0.000 \\
    78    & 0.0000 & 0.000 \\
    121   & 0.0003 & 1.74x10$^{3}$  \\
    218   & 0.0246 & 2.32x10$^{5}$  \\
    456   & 0.2951 & 4.13x10$^{6}$ \\ \cline{1-3}
    \multicolumn{2}{l}{Total power flux [$\mu$W/m$^2$]:} & 4.2 \\ \cline{1-3}
    \end{tabular}
    \begin{flushleft}
    * These are the same flux measurements as those shown in Figure \ref{fig:JEDIFlux}.
    
    $\dagger$ X-ray efficiency values in Appendix \ref{app:Oxy} or \ref{app:Sul}. These values are the sum of O$^{6+}$ and O$^{7+}$ from both charge exchange and direct excitation for an exit angle of 80$^{\circ}$ in atmosphere 1.
    \end{flushleft}
    \label{tab:JEDIFlux}
\end{table}

The total X-ray production shown in Table \ref{tab:JEDIFlux} is only about 7$\%$ percent higher than what is shown in Figure \ref{fig:JEDISpec}, where the power flux was found by integrating over every photon energy.
This exact same process can be used for sulfur, but in this example sulfur emission is much less than oxygen.
